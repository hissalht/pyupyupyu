%parler des lib quand on s'en sert
%parler des standby liste




\documentclass[a4paper, 12pt]{report}

\usepackage[utf8]{inputenc}
\usepackage[francais]{babel}
\usepackage{graphicx}
\usepackage{bookman}
\usepackage{titlesec}
\titleformat{\chapter}[hang]{\bf\huge}{\thechapter}{2pc}{}
\renewcommand\thesection{\Roman{section}}
\renewcommand\thechapter{\Roman{chapter}}

\title{Rapport Manic Shooter}
\author{Adrien \bsc{Boutigny}, Tanel \bsc{Creusier} et Matthieu \bsc{Dechipre}}
\date{L2 Informatique 2015 2016}

\pagestyle{headings}

\begin{document}

\maketitle

\tableofcontents

\chapter{Introduction}
	\section{Contexte}
Le shooter est un domaine très vaste du jeu vidéo utilisant des codes et des
logiques bien à lui. Lors de ce semestre, nous avons dû travailler sur un
sous-genre de jeu de tir, le Manic Shooter. Ce type de jeu impose plusieurs
choses, tout d'abord, un nombre de projectiles ahurissant et en règle général
une dextérité importante de la part du joueur pour ce mouvoir dans ce maelström
de balles, de tirs et de lasers en tout genre.

Ci-dessous, voici l'un des piliers du genre, DoDonPachi:

\begin{figure}[!ht]
\centering
\includegraphics[scale= 0.4]{images/dodonpachi.jpg}
\caption{DoDonPachi}
\end{figure}

    \section{Objectifs du projet}
Dans cette application que nous avons réalisé, il fallait que le joueur
puisse passer aux commandes d'un vaisseau capable de se déplacer, de tirer et
d'utiliser divers bonus et items récupérables au cours de la partie. \newline

\noindent De plus, il fallait également gérer les ennemis. Etudier des manières de se
déplacer, de tirer pour ces dernier. Il fallait également réfléchir aux
méthodes d'apparition possibles et trouver comment faire en sorte que les
projectiles tirés par le joueur touche les vaisseaux ennemis et leur fasse
perdre des points de vie.\newline

\noindent L'idée était donc de créer un jeu de tir utilisant le clavier pour les
déplacement, dans un premier temps, et le tir via une librairie python prévu
spécialement pour  tout ce qui est multimédia (événementiel, vidéo \ldots) Dans
un second, il fallait peut-être imaginer une manière plus dynamique de déplacer
le vaisseau du joueur, pour ce type de shooter, la précision était de rigueur.
Aussi fallait-il permettre au joueur de modifier soi-même ses touches dans un
second temps. \newline

\noindent Le premier semestre s'est vu consacré à la création des fonctions principales
du programme, comme le déplacement, l'affichage des vaisseaux \ldots \newline

\noindent Dans une première partie, nous expliquerons donc ce que nous avons décidé de
mettre en place pour ce jeu, toutes les fonctionnalités que nous avons prévu
d'implémenter.

\chapter{Cahier des charges}
    \section{Fonctionnalités à réaliser}
Pour réaliser ce Manic Shooter, nous avons commencé par faire un brain
storming pour déterminer tout ce dont nous aurions besoin pour faire ce jeu et
ce que nous devrions implémenter. Ainsi, nous avons obtenu les fonctionnalités
suivante: \newline

\begin{itemize}
    \item un vaisseau joueur pouvant tirer et se déplacer
    \item un système de score
    \item un menu
    \item un système de musique
    \item différents types d'ennemis eux aussi capables de se déplacer et de tirer
    \item gestion des collisions entre les ennemis, les projectiles et le joueur
    \item des powers-ups
    \item un menu et un système de configuration des touches
    \item un système de génération de niveaux \newline
\end{itemize}

\noindent Nous insérons d'implémenter toutes les fonctionnalités ci-dessus tout le long
de cette année en épurant le gros du travail pendant le premier semestre. Lors
du second, nous implémenterons diverses fonctionnalités supplémentaires et nous
optimiserons le programme afin qu'il soit plus performant.

\chapter{Fonctionnement}
	%\section{Langages, logiciels et bibliothèques utilisées}

%Pour créer ce shooter, nous avons utilisés plusieurs choses. Tout
%d'abord, nous avons implémenté tout le programme en Python 3.4.\newline

%\noindent Les librairies que nous avons utilisés sont pysdl2 et pyglet:

%\subsection{Pysdl2}

%\begin{quotation}
    %SDL (Simple DirectMedia Layer) est une librairie utilisée pour créer des
    %applications multimédias en deux dimensions comme les jeux vidéos, les
    %démos graphiques, les émulateurs, etc. [\ldots]

    %\noindent Cette bibliothèque est disponible sous Linux, Windows, Windows CE, BeOS,
    %Mac OS, Mac OS X, FreeBSD, NetBSD, OpenBSD, BSD/OS, Solaris, IRIX et QNX\@.
    %Elle fonctionne aussi sous quelques systèmes embarqués comme des consoles
    %de jeu portable.
    %\indent \indent Wikipédia
%\end{quotation}

%\noindent Cette librairie est donc parfaitement adapté à la création d'un jeu vidéo comme
%notre shooter, de plus, il est a noter que Pygame a été crée à partir de cette
%librairie.

%\subsection{Pyglet}

%Nous avons utilisé Pyglet simplement pour permettre de gérer le flux
%audio, ce que la librairie pysdl2 ne permet pas nativement, en tout cas pas
%sans un travail de fond. Aussi, nous avons trouvé une solution temporaire via
%Pyglet pour ajouter une ambiance sonore au jeu.\newline

%\noindent Cette librairie est, tout comme Pysdl2, une librairie multimédia qui fait
%quasiment le même travail que cette dernière. Aussi, elle est trop volumineuse
%pour une simple librairie audio. Aussi, nous espérons faire en sorte de faire
%fonctionner la librairie audio de Pysdl2 le long du second semestre et ainsi
%remplacer Pyglet.

%\noindent Des exemples de fonctionnement de la partie audio de SDL sont déjà disponibles
%sur le dépot.

\section{Les collisions}


Les collisions sont au coeur de notre projets. En effet, bien que la gestion de
celles-ci soit d'une simplicité enfantine, elle n'en demeure pas moins utilisée
par quasiment tout les modules, et en très grand nombre.

Pour la gestion des collisions, on a le choix entre 2 méthodes différentes
ayant chacune ses avantages et ses défauts:
\begin{itemize}
    \item Soit on utilise des formes géomètriques (des rectangles) et on teste les collisions
        en vérifiant des inéquations ``simples''. Méthode simple à mettre en
        place et rapide.
    \item Soit on utilise des ``masques de collisions'', c'est à dire une image
        ``invisible'' qui sera superposé à l'element concerné. L'image suit un
        code couleur indiquant quelles sont les parties solides de l'element.
        Par exemple, un pixel blanc indique que l'element est solide et un
        pixel noir qu'il n'y a rien à cet endroit. Pour savoir si deux élements
        rentrent en collision, on teste si des pixels ``blancs'' se
        superposent.
\end{itemize}


Pour l'instant nous utilisont uniquement des collisions à base de rectangles.
Le coût peu élevé de l'algorithme nous arrange car nous avons parfois des
centaines de collisions à tester à chaque frame.

%inserer ici une image d'un vaisseau ennemi avec un rectangle representant sa hitbox


%\noindent L'algorithme de collisions est simple car, comme nous l'avons dis
%précédemment, il s'agira de détecter énormément de collisions. Nous
%n'utiliserons donc pas d'algorithmes détectant les collisions au pixel près car
%cela serait trop coûteux.\newline

\noindent Cet algorithme simple peut s'exprimer facilement avec l'illustration tirée du
site développez.com suivante. Ici, le bleu est avant la ligne jaune minimum du
rouge, bleu et rouge ne sont pas en collision et le jaune est avant le maximum
du rouge, ils sont en collision:

\begin{figure}[!ht]
\centering
\includegraphics[scale= 0.8]{images/collision-boites.png}
\caption{Principe de collisions simples}
\end{figure}

\section{Le monde}

Pour mettre en place tous les éléments nécessaires au jeu, il a d'abord
fallu définir un décor dans lequel les faire évoluer. Pour ce faire, nous avons
créé une fenêtre via la librairie Pysdl dans un premier temps. Ensuite, il a
fallu mettre en place le scrolling vertical que nous voulions pour notre
shooter. Nous avons donc utilisé une ``caméra'' qui se déplace constamment vers
le haut et qui fait défiler le fond du jeu.\newline

\noindent Les coordonnées en y de la caméra ne font donc que diminuer (on se situe dans
un repère inversé). La carte est pré-établie au lancement du niveau, les
ennemis aussi. En remontant, on va donc vérifier si les coordonnées en y des
ennemis coïncident avec celles de la caméra grâce à notre fonction de
collision. Si la caméra entre en collision avec les vaisseaux ennemis contenus
dans une liste créée au tout début du programme alors on les active et on les
affiche.\newline

\begin{figure}[!ht]
\centering
\includegraphics[scale= 0.4]{images/schema.png}
\caption{Principe de la caméra}
\end{figure}


\noindent Le monde contient également le vaisseau du joueur qui doit être manipulé par
l'utilisateur et qui peut dans cette première version tirer et bouger. Comme
dit précédemment, les ennemis actifs et passifs (ceux qui ne sont pas encore
arrivés à l'écran) sont contenus dans le monde, ainsi que tous les projectiles
qui ont été tiré. \newline

\noindent Les méthodes des vaisseaux ennemis et du joueur interagissent donc avec le
monde et ce qu'il contient lors des collisions par exemple ou encore lorsqu'ils
tirent.


\section{Les vaisseaux}

Nous avons implémenté les vaisseaux de manière à ce qu'il se base tout
sur un modèle commun. Ce modèle commun définit des paramètres par défaut d'un
vaisseau quelconque, les points qu'il attribue au joueur lorsqu'il est détruit
ou ses points de vie par exemple.\newline

\noindent A partir de ce squelette de vaisseau, nous avons créé des sous classes pour
tout les autres vaisseaux, celui du joueur d'un côté, avec ses méthodes
propres, et ceux des ennemis de l'autre. Chaque ennemi a ses méthodes et chaque
ennemi est spécifique. Certains se contenteront de passer sans rien faire,
d'autres pourront tirer en visant le joueur ou d'autres encore pourront se
déplacer en suivant un chemin complexe (cercle, sinusoïde \ldots)\newline

\begin{figure}[!ht]
	\centering
	\begin{minipage}[t]{.49\linewidth}
		\centering
		\includegraphics[height=200pt, width=180pt]{images/johncena.png}
		\caption{Ennemi tirant beaucoup de projectiles}
	\end{minipage}
	\begin{minipage}[t]{.49\linewidth}
		\centering
		\includegraphics[height=200pt, width=180pt]{images/poissons.png}
		\caption{Ennemis visant le joueur}
	\end{minipage}
\end{figure}

\noindent De ce fait, nous avons un module plutôt conséquent contenant tous les types
de vaisseaux différents.

\chapter{Réalisation}
\section{Répartition des tâches}

Après avoir fait la liste de tout ce dont nous avions besoin pour créer ce jeu,
nous avons réparti les tâches entre les différents membres du groupe. Chacun
s'est lancé dans ce qu'il voulait faire. En effet, nous pouvions développer
chacun de notre côté assez facilement. Nous avons pu implémenter simultanément
la boucle principale gérant le jeu, le système de collisions et les différents
vaisseaux.\newline

\noindent Par la suite, il est facile de travailler sur les modules externes
comme les bonus, le système de configuration des touches ou encore d'autres
types d'ennemis et ensuite les greffer au code déjà existant.

\chapter{Retours sur l'application et apports possibles}

Nous avons pu implémenter pour l'instant une bonne partie des éléments du
projet, à savoir des vaisseaux différents, un vaisseau joueur pouvant tirer et
se mouvoir. Nous avons également mis en place un menu de départ qui reste
encore à fignoler et le système de collisions. Nous avons également la
possibilité de mettre le jeu en pause. \newline

\noindent Les ennemis ont quelques patterns de mouvements et de tirs originaux
mais pour l'instant nous n'avons pas de choses vraiment complexes comme des
laser continus ou encore des boucliers. Nous essaierons donc de mettre ces
éléments en place lors du second semestre ainsi que le générateur de niveaux et
la possibilité de modifier sa configuration de contrôle. \newline

\noindent Nous mettrons aussi en place divers bonus parmi ceux\newline cités
précédemment comme des laser continu, des boucliers, des accélérations de tir
ou encore des explosions à courte portée. Il faudra également remplacer la
gestion du son de la librairie Pyglet par la partie audio de SDL\@.

\chapter{Suppléments}
	\section{Sources}
\noindent https://www.libsdl.org/ \newline
https://www.python.org/download/releases/3.4.0/ \newline
https://bitbucket.org/pyglet/pyglet/wiki/Home

	\section{Documentation}
\noindent http://www.developpez.com/ \newline
https://openclassrooms.com/ \newline
https://pysdl2.readthedocs.org/en/latest/ \newline
https://pyglet.readthedocs.org/en/pyglet-1.2-maintenance/

\end{document}
